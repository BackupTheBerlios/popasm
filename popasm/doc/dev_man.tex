\newcommand{\popasm}{\emph{PopAsm}}
\newcommand{\nasm}{\emph{NASM}}
\newcommand{\tasm}{\emph{TASM}}

\documentclass[a4paper,draft,12pt]{book}
\title{\popasm\ --- The Popular Assembler\\Developer's Manual}
\author{Helcio B. de Mello}
\date{\today}

\begin{document}

\begin{titlepage}
\maketitle
\end{titlepage}

\pagenumbering{roman}

\bf{NOTE:} This is but a draft of the \popasm\ Developer's Manual.
The contents of this document may change as needed, until the release
of the first official full version. Comments on English mistakes
or contents are welcome, given the author's native language is not
English. Please see appendix \ref{CONTACT} for contacting info.

\chapter*{About this document}
This document is part of \popasm, the Popular Assembler Project.
It has been totally written in \LaTeX\ by
\popasm\ author himself.

At the time of this writing, \popasm\ had two manuals:

\begin{itemize}
\item{User's Manual} --- Aimed to those who wish to \emph{use}
\popasm, regardless the way it was developed and its internal
implementation. This document also concerns about instalation
procedures.
\item{Programmer's Guide} --- This document. Documents \popasm\
source code in detail. It is recommended to those who want to
read, understand and/or modify \popasm\ sources.
\end{itemize}

Please check which manual is the one intended for your needs.

\chapter*{About the author}
H\'elcio Mello was born in Brazil, in 1979. He has a degree in Computer
Engineering from Federal University of
Espirito Santo -- ES -- Brazil\cite{UFES}, and is curently taking
Computer Science in Pontificial Catholic
University in Rio de Janeiro -- RJ -- Brazil\cite{PUC}. He is expected
to be a Master of Science in 2004.

He joined \emph{Sourceforge}\cite{SF} in December 2001, and then
registered \popasm\ as a \emph{SourceForge}
project. The project was finally open to the Free Software community.
In July 2002, he joined \emph{Advogato}
\cite{ADV,ADVPRSN} and \emph{FreshMeat}\cite{FRESH}

\chapter*{Acknowledgements}
\popasm\ is the result of many months of work. Despite being the
sole author of this project, I would like to
thank everybody that somehow contributed to this project. Among those
are the hundreds of people who have already
visited \popasm\ home page, talked about it with friends, mailed
me for comments, and so on.

I would also like to thank the ones who use \popasm\ and trust it.
I know I can always count on you for
feedback about this project. \emph{Sourceforge} support was crucial, as
it hosts \popasm\ CVS repository,
home page, and so on. Thank you very much.

Special thanks go to my first university, UFES\cite{UFES}, where
I learned the bulk of my computering skills, and
to those classmates and teachers who believed \popasm\ project,
such as professor and PhD S\'ergio Freitas%
\cite{FREITAS}, who decided to adopt \popasm\ as the assembler for
his graduate classes about assembly language.

Last, but not least, I would like to thank my family, friends and God
for what I am now and what I have done so far.



\tableofcontents
\newpage
\listoftables
\newpage
\listoffigures
\newpage

\pagenumbering{arabic}
\part{Getting Started}
This initial part starts with an introductory chapter which gives an
overall introduction of how \popasm\ code is organized, so the reader
can identify which of the chapters ahead will be of his interest.

The remaining chapters deal with the main classes used by \popasm\ 
to perform all of its functions. They are intended to those who
wish to understand how things work at the highest level of abstraction,
ignoring implementation details of the base classes (eg.: how two
numbers are added is not a concern in this part).

\chapter{Introduction}
As many projects, \popasm\ code is distributed in many source files
due to the clear advantages of modular design. The objective of this
chapter is to offer the reader an overview of how \popasm\ code is
organized, that is, what modules have been defined, which functionality
each one offers, and how they interact with one another.

Everything starts at main() function in main.cpp file. Its job is to
start \popasm\ up, parse command line options and so on (details in
Chapter \ref{chapmain}). After main() is done with this initialization
it calls the methods for assembling the input files.

Note that \popasm\ source defines an Assembler class (see Chapter
\ref{chapasm}) which carries out the assembling task via its
AssembleFile() method. This method can be seen as the ``root'' of
\popasm\ execution, for it calls the other vital modules of the
assembler.

One of such modules is the parser (described in Chapter \ref{chapparser}),
which calls the lexical analyzer (see Chapter \ref{chaplex}) to read
the keywords, expressions, etc. from the input files. Such lexical
items are then checked to verify whether or not they follow \popasm\ 
syntax. If they do, the parser then converts them into arguments
(such as memory references) and give them to the instruction (or
directive) found in the current line.

The Instruction class (explained in Chapter \ref{chapinst}) gets such
arguments and performs additional checking (eg. type checking). If
they succeed the instruction is assembled and the resulting encoding
is returned to the parser.

The paragraphs above rawly descrive how the assembly process works.
However many details have been intentionally ommited in order to
make the explanation simple. Such details are fully covered in other
chapters.

\chapter{What main() does\label{chapmain}}
To be written.

\chapter{The assembler classes\label{chapasm}}
One of \popasm\ greatest concerns is compatibility with existing code.
Such code was written according to syntatic rules due to other
assemblers. Most of those rules can be gathered together in a single
assembler (like \popasm\ does), but some of them conflict with one
another. For example, in \tasm\ a variable name stands for its contents
while in \nasm\ it returns its offset.

Due to such conflicts, command line options have been created to change
\popasm\ configuration. Nevertheless, there are differences among
assemblers which are better tackled by emulating the target assembler
(for example, operator precedence is slightly different between \popasm\ 
and \nasm.).

In order to do that, a base class from which all assemblers to be emulated
derive has been defined. It contains all methods required to perform
the assembly task. The emulated assemblers then inherit from the base
class and overrides the necessary methods to achieve compatibility.
The next sections describe the Assembler base class and PopAsm, the
only supported assembler at the time of this writing.

\section{Assembler base class}
The definition of this class can be found in asmer.h file. It contains four
integer members:

\begin{itemize}
\item InitialMode --- contains the mode (16 or 32 bits) the assembler must
be set to before assembling the source file. In other words, this is the
default operating mode the source will be assembled at, unless explicitly
changed (eg. via a BITS directive);

\item CurrentMode --- Keeps track of which mode the assembler is operating
at. This is the member changed by the BITS directive.

\item CurrentPass --- This member is used to store which pass the assembler
is in. It may be necessary to perform several passes to assemble a single file.

\item CurrentOffset --- stores the memory address the next instruction or
variable will occupy in memory.
\end{itemize}

The methods GetCurrentMode(), SetCurrentMode() and GetCurrentOffset() allow
the interaction with those members.

The only constructor for this class takes the initial mode as its argument.
It initializes the InitialMode member to this value and sets CurrentPass
to zero. Note that CurrentMode and CurrentOffset do not need to be
initialized because the assembler will do that properly at the beginning of
each pass.

The other methods are PerformPass() and AssembleFile(). The latter takes an
input file (InputFile is described in Chapter \ref{chapinp}) as argument and
assembles it. This is done by calling PerformPass() as many times as needed.

PerformPass(), in turn, performs a single pass through the input file and
returns true if another pass is needed or false if assembly success or failure
has already been determined. This method resets the CurrentOffset and CurrentMode
members and parses the lines from the input file, one by one (by means of
Parser::ParseLine() method. See chapter \ref{chapparser} for details). After
parsing a line, this method updates the CurrentOffset accordingly.

\section{PopAsm class}
This is currently the only class that inherits from Assembler, defined in
files popasm.cpp and popasm.h. The constructor takes the initial mode as
its only argument. It also initializes the operator, encloser, instruction
and directive tables. The destructor is responsible for releasing all memory
from the operator and encloser tables (the other tables use static memory and
thus do not need to be destroyed).

All the heavy work is performed in the base class by its default AssemblyFile()
and PerformPass() methods.

\chapter{The Parser\label{chapparser}}

\chapter{Instructions\label{chapinst}}








\chapter{Lexical Analyzer\label{chaplex}}

\chapter{Definitions\label{chaprefs}}

\chapter{Numbers\label{chapnums}}

\chapter{Input Files\label{chapinp}}

\chapter{Hash Tables\label{chaphash}}

\chapter{Numbers Again\label{chapnumsagain}}

\chapter{Operators\label{chapop}}

\chapter{Registers\label{chapregs}}




\appendix

\chapter{Contacting info\label{CONTACT}}
Up to date info about contacting \popasm\ team can be found at \popasm\ 
homepage, at http://popasm.sourceforge.net.

\begin{thebibliography}{12}
\bibitem{UFES} Federal University of Esp\'{\i}rito Santo -- ES -- Brazil.
http://www.ufes.br
\bibitem{PUC} Pontificial Catholic University of Rio de Janeiro -- RJ
-- Brazil. http://www.puc-rio.br
\bibitem{SF} SourceForge. http://sorceforge.net
\bibitem{ADV}Advogato. http://www.advogato.com
\bibitem{ADVPRSN}H\'elcio Mello's personal page at Advogato.
http://www.advogato.com/person/helcio
\bibitem{FRESH}FreshMeat. http://www.freahmeat.net
\bibitem{FREITAS}Professor PhD S\'ergio A. A. Freitas.
http://www.inf.ufes.br/\~{}sergio.
\bibitem{AUTOCONF}GNU Autoconf.
http://www.gnu.org/software/autoconf/autoconf.html
\bibitem{AUTOMAKE}GNU Automake.
http://www.gnu.org/software/automake/automake.html
\bibitem{MAKE}GNU Make. http://www.gnu.org/software/make/make.html
\bibitem{FPUMAN}FPU Programming on x86 FPU's
\end{thebibliography}

\end{document}
