\documentclass[a4paper,12pt]{book}
\title{PopAsm --- The Popular Assembler\\User's Manual}
\author{Helcio B. de Mello}
\date{\today}

\begin{document}
\maketitle

\chapter*{About this document}
This document is part of \emph{PopAsm, the Popular Assembler Project}. It has been totally written in \LaTeX\ by
\emph{PopAsm} author himself.

At the time of this writing, \emph{PopAsm} had two manuals:

\begin{itemize}
	\item{User's Manual} --- this document. It is aimed to those who wish to use \emph{PopAsm}, regardless the way
		it was developed and its internal implementation. This document also concerns about instalation procedures.
	\item{Programmer's Guide} --- documents \emph{PopAsm} source code in detail. It is recommended to those who
		want to read, understand, compile and/or modify \emph{PopAsm} sources.
\end{itemize}

Please check which manual is the one intended for your needs.

\chapter*{About the author}
H\'elcio Mello was born in Brazil, in 1979. He has a degree in Computer Engineering from Federal University of
Espirito Santo -- ES -- Brazil\cite{UFES}, and is curently taking Computer Science in Pontificial Catholic
University in Rio de Janeiro -- RJ -- Brazil\cite{PUC}. It is expected that he will be a Master of Science in 2004.

He joined \emph{Sourceforge}\cite{SF} in December 2001, and then registered \emph{PopAsm} as a \emph{SourceForge}
project. The project was finally open to the Free Software community. In July 2002, he joined \emph{Advogato}
\cite{ADV}\cite{ADVPRSN} and \emph{FreshMeat}\cite{FRESH}

\chapter*{Acknowledgements}
\emph{PopAsm} is the result of many months of work. Despite being the sole author of this project, I would like to
thank everybody that somehow contributed to this project. Among those are the hundreds of people who have already
visited \emph{PopAsm} home page, talked about it with friends, mailed me for comments, and so on.

I would also like to thank the ones who use \emph{PopAsm} and trust it. I know I can always count on you for
feedback about this project. \emph{Sourceforge} support was crucial, as it hosts \emph{PopAsm} CVS repository,
home page, and so on. Thank you very much.

Special thanks go to my first university, UFES\cite{UFES}, where I learned the bulk of my computering skills, and
to those classmates and teachers who believed \emph{PopAsm} project, such as professor and PhD S\'ergio Freitas%
\cite{FREITAS}, who decided to adopt \emph{PopAsm} as the assembler for his graduate classes about assembly
language.

Last, but not least, I would like to thank my family, friends and God for what I am now and what I have done so far.

\chapter{Introduction}
PopAsm stands for ``Popular Assembler''. It is an assembler, that is, and assembly language compiler\footnote{Technically
speaking, assemblers are not compilers. They translate lines of code into machine language in a one-to-one basis. Compilers
translate each line of code into several machine instructions. However, from this point on, this document will use both
terms (compiler and assembler) indistinctly.}. Its objective is to convert human readable code to machine instructions.
These instructions will then be either executed as binary programs or linked with other modules (possibly written in
other languages) to yield a computer program.

Many assemblers already exist. Some are free and open source, others are not. Some will offer you features that others
will not. Some will be suited for your needs, but others will not. \emph{PopAsm} was designed to gather in a single
assembler the best features of the existing assemblers, yet adding its own improvements and remaining compatible with
existing code as well. As a result, most of your legacy code can be compiled under \emph{PopAsm} without any modifications
at all.

Besides the benefits discussed so far, \emph{PopAsm} is a free open source project written in ANSI C++, which means
that anyone can read its source code, modify, and compile it anywhere an ANSI C++ compiler is avaiable. Its peculiar
features make it suited for nearly any assembly programming project:

\begin{itemize}
	\item{Huge numbers internal representation} allows assembly-time expression evaluation in both integer and floating
		point format without any practical limit;
   \item{Smart default options} make your code cleaner, without the redundancy demanded by some other assemblers;
	\item{Top flexibility} gives the developers the choice to use the infamous ``red tape'' present in some assemblers
		or just write the good old raw assembly code;
	\item{And more...}
\end{itemize}

Due to the reasons discussed above, this assembler was named the ``Popular Assembler''.

\begin{thebibliography}{9}
\bibitem{UFES} Federal University of Esp\'irito Santo -- ES -- Brazil. http://www.ufes.br
\bibitem{PUC} Pontificial Catholic University of Rio de Janeiro -- RJ -- Brazil. http://www.puc-rio.br
\bibitem{SF} Pontificial Catholic University of Rio de Janeiro -- RJ -- Brazil. http://www.puc-rio.br
\bibitem{ADV}Advogato. http://www.advogato.com
\bibitem{ADVPRSN}H\'elcio Mello's personal page at Advogato. http://www.advogato.com/person/helcio
\bibitem{FRESH}FreshMeat. http://www.freahmeat.net
\bibitem{FREITAS}Professor PhD S\'ergio A. A. Freitas. http://www.inf.ufes.br/~sergio.

\end{thebibliography}

\end{document}
