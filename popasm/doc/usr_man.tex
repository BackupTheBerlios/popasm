\documentclass[a4paper,12pt]{book}
\def\thebibliographyname{References}
\title{PopAsm --- The Popular Assembler\\User's Manual}
\author{Helcio B. de Mello}
\date{\today}

\newcommand{\popasm}{\emph{PopAsm}}
\newcommand{\nasm}{\emph{NASM}}
\newcommand{\tasm}{\emph{TASM}}

\begin{document}
\maketitle

\chapter*{About this document}
This document is part of \popasm, the Popular Assembler Project.
It has been totally written in \LaTeX\ by
\popasm\ author himself.

At the time of this writing, \popasm\ had two manuals:

\begin{itemize}
\item{User's Manual} --- this document. It is aimed to those who wish
to use \popasm, regardless the way
it was developed and its internal implementation. This document also
concerns about instalation procedures.
\item{Programmer's Guide} --- documents \popasm\ source code in
detail. It is recommended to those who
want to read, understand, compile and/or modify \popasm\ sources.
\end{itemize}

Please check which manual is the one intended for your needs.

\chapter*{About the author}
H\'elcio Mello was born in Brazil, in 1979. He has a degree in Computer
Engineering from Federal University of
Espirito Santo -- ES -- Brazil\cite{UFES}, and is curently taking
Computer Science in Pontificial Catholic
University in Rio de Janeiro -- RJ -- Brazil\cite{PUC}. It is expected
that he will be a Master of Science in 2004.

He joined \emph{Sourceforge}\cite{SF} in December 2001, and then
registered \popasm\ as a \emph{SourceForge}
project. The project was finally open to the Free Software community.
In July 2002, he joined \emph{Advogato}
\cite{ADV}\cite{ADVPRSN} and \emph{FreshMeat}\cite{FRESH}

\chapter*{Acknowledgements}
\popasm\ is the result of many months of work. Despite being the
sole author of this project, I would like to
thank everybody that somehow contributed to this project. Among those
are the hundreds of people who have already
visited \popasm\ home page, talked about it with friends, mailed
me for comments, and so on.

I would also like to thank the ones who use \popasm\ and trust it.
I know I can always count on you for
feedback about this project. \emph{Sourceforge} support was crucial, as
it hosts \popasm\ CVS repository,
home page, and so on. Thank you very much.

Special thanks go to my first university, UFES\cite{UFES}, where I
learned the bulk of my computering skills, and
to those classmates and teachers who believed \popasm\ project,
such as professor and PhD S\'ergio Freitas%
\cite{FREITAS}, who decided to adopt \popasm\ as the assembler for
his graduate classes about assembly language, as soon as it is
fully functional.

Last, but not least, I would like to thank my family, friends and God
for what I am now and what I have done so far.

\tableofcontents
\newpage
\listoftables
\newpage

\chapter{Introduction}
\popasm\ stands for ``Popular Assembler''. It is an assembler,
that is, and assembly language compiler\footnote{Technically
speaking, assemblers are not compilers. They translate lines of code
into machine language in a one-to-one basis. Compilers
translate each line of code into several machine instructions. However,
from this point on, this document will use both
terms (compiler and assembler) indistinctly.}. Its objective is to
convert human readable code to machine instructions.
These instructions will then be either executed as binary programs or
linked with other modules (possibly written in
other languages) to yield a computer program.

Many assemblers already exist. Some are free and open source, others
are not. Some will offer you features that others
will not. Some will be suited for your needs, but others will not.
\popasm\ was designed to gather in a single
assembler the best features of the existing assemblers, yet adding its
own improvements and remaining compatible with
existing code as well. As a result, most of your legacy code can be
compiled under \popasm\ without any modifications
at all.

Besides the benefits discussed so far, \popasm\ is a free open
source project written in ANSI C++, which means
that anyone can read its source code, modify, and compile it anywhere
an ANSI C++ compiler is avaiable. Its peculiar
features make it suited for nearly any assembly programming project:

\begin{itemize}
\item{Huge numbers internal representation} allows assembly-time
expression evaluation in both integer and floating
point format without any practical limit;
   \item{Smart default options} make your code cleaner, without the
redundancy demanded by some other assemblers;
\item{Top flexibility} gives the developers the choice to use the
infamous ``red tape'' present in some assemblers
or just write the good old raw assembly code;
\item{And more...}
\end{itemize}

Due to the reasons discussed above, this assembler was named the
``Popular Assembler''.

\chapter{Compiling and Installing \popasm}
This chapter explains how to compile and install \popasm\ under
Linux or other UNIX-like platforms and DOS/Windows. If you have
already done that you may safely skip to the next chapter.

\section{UNIX environments}

This section describes compiling \& installing info under UNIX-like
operating systems (such as Linux).

\subsection{Compiling PopAsm sources\label{COMPUNIX}}
In order to compile \popasm\ sources, you will need an ANSI C++
compiler and its standard libraries\footnote{For the curious, I
use egcs 2.91.66}. You will also need \popasm\ source code, which
you should already have. If not, please go to
\emph{http://popasm.sourceforge.net/} and download it. You will
get an archive containing the source code to be built.

Unpack it anywhere you like, using the appropriate software (e.g.
if you downloaded a .tar.gz file, you should run
\emph{tar xvzf popasm-x.y.z.tar.gz}, where $x$, $y$ and $z$ are
the version numbers of the package you got). At this point, a
directory containing the source code will be created.

Now, you should enter the directory containing the sources. e.g.
\emph{/tmp/popasm-0.0.1}. Please do not mistake it for something
like \emph{/tmp/popasm-0.0.1/src}. You should be in \popasm
source code root directory, as in the example above.

\popasm\ relies on \emph{Autoconf}\cite{AUTOCONF} and
\emph{Automake}\cite{AUTOMAKE} to probe your system for the
necessary resources and generate the resulting \emph{Makefile}.
This file will tell the \emph{Make}\cite{MAKE} utility how to
compile \popasm\ sources. First, type

\begin{verbatim}
./configure
\end{verbatim}

to perform the necessary checks. If everything is ok, you should have a
\emph{Makefile}. The next step is to compile the sources. Just type

\begin{verbatim}
make
\end{verbatim}

and the \emph{Make} utility will do the rest, but might take a few
minutes. There should be no warnings and no errors. If you got any,
please check whether your compiler is ANSI compliant. If it is,
please let me know (contacting information can be found in this
document).

\subsection{Installing \popasm}

There is basically two ways of installing \popasm: either from
the source code or from a RPM file.

\subsubsection{Installing from Source Code}
You should compile \popasm\ as in section \ref{COMPUNIX}. Remain
in the directory you built \popasm\ and issue the command (you will
need to be \emph{root} to do that):

\begin{verbatim}
make install
\end{verbatim}

This will install the binary file, documentation, etc.

\subsubsection{Installing from RPM files}
The easiest way to install \popasm\ is to use RPM files. Get one from
\emph{http://popasm.sourceforge.net/} if you have not done so yet.
Login as \emph{root} and type:

\begin{verbatim}
rpm -ivh popasm-x.y.z.rpm
\end{verbatim}

replacing $x$, $y$ and $z$ for the appropriate version numbers. That's
it, you're done.

\section{DOS and Windows environments}
To be written.

\subsection{Compiling PopAsm sources\label{COMPDOS}}
To be written.

\subsection{Installing \popasm}
To be written.

\chapter{Running PopAsm\label{CMDLINE}}

This chapter tackles the usage of \popasm\ under command line. Arguments,
options and environment variables are discussed in detail.

To be written.

\chapter{PopAsm syntax}

One of \popasm's major goals is compatibility with existing code.
\popasm\ syntax is very similar to the ones supported by \tasm\ and
\nasm. This chapter discusses \popasm\ basic syntax. The next chapters
go into more detail in each syntax component (numbers, registers, etc.).

As stated in chapter \ref{CMDLINE}, \popasm\ takes one or more source
files as inputs, each one containing assembly-language statements. These
statements, in turn, have the following general form:

\begin{verbatim}
label:          CMD     arguments               ; comments
\end{verbatim}

where:

\begin{itemize}
\item{\emph{label}} is an identifier that marks an offset into the current
   memory segment, so it can be referenced elsewhere in the code. The colon
   (:) after the label is optional, but recommended, as discussed in section
   \ref{NOTELABEL}.
\item{\emph{CMD}} is either a x86 instruction (such as MOV, ADD, etc.) or a
   \popasm\ internal command (ORG, DB, etc.). Command names are
   case-insensitive (MOV, mov and Mov are all accepted).
\item{\emph{arguments}} is a comma-separated list of arguments for the CMD
   command used in this line. An argument is any sort of \emph{expression}
   (see chapter \ref{EXPR}). Some commands neither require nor accept arguments.
\item{\emph{comments}} is whatever you want to write to make your code easier
   to read and maintain. Comments are placed after a semicolon, and
   \popasm\ ignores them altogether.
\end{itemize}

For example:

\begin{verbatim}
MULTIPLY:       IMUL    CX,BX,5                 ; CX = BX * 5
\end{verbatim}

\noindent is a valid statement. MULTIPLY is a label (followed by the
optional colon), IMUL is a x86 instruction, and CX, BX and 5 are
arguments for the IMUL instruction. The string following the semicolon
is a comment, and is thus ignored.

None of the items above are mandatory, except that you cannot specify
arguments for a command without issuing it first. So,

\begin{verbatim}
                MOV     AX,BX
\end{verbatim}

\noindent is a valid statement, but

\begin{verbatim}
                AX,BX
\end{verbatim}

\noindent is obviously not.

\popasm\ imposes no tabbing restrictions; white spaces and tabs are
ignored. This document uses a 1--16--24--48 tabbing scheme (that is,
labels at column 1, commands at column 16, arguments at column 24 and
comments at column 48).

\section{Notes on labels\label{NOTELABEL}}

Valid labels are strings of letters (a--z or A--Z), digits (0--9 )or any
of the following characters:

\begin{center}
\_ @ \$ ?
\end{center}

\noindent except that the first character cannot be a digit, otherwise the
label would be treated as a number, as discussed in chapter \ref{NUMCHAP}.

\popasm\ is by default case-insensitive\footnote{Due to compatibility with
\tasm.}. Such behavior may be changed by command line options and environment
variables (see chapter \ref{CMDLINE}). This document uses uppercase letters.

As already mentioned, the use of a colon after a label is optional,
but recommended. For example, suppose you issue a LODSB command, but
mistype it for LOADSB instead. As a result, the mistyped command would
be treated as an ordinary label, and thus not compiled properly. \nasm\ 
calls these ``orphan labels'', and offers a command line option to
warn you about that.

But let's check the code below:

\begin{verbatim}
I_LOVE_USING_LONG_LABELS_THROUGHOUT_MY_CODE
                LOADSB
                OR      AL,AL
                JNZ     I_LOVE_USING_LONG_LABELS_THROUGHOUT_MY_CODE
\end{verbatim}

The first line is a rather long label, and the developer decided
to place it alone in its line of code. The next line is the mistyped
command. If the developer decides to enable the orphan-label warning,
an annoying message will always bother him, saying the long label
is orphan. Using the colon, this problem is solved.

Another good reason to use the optional colon is to have an additional
mark to make it clear, to the ones reading your code, that the label is,
indeed, a label, instead of, say, a macro name. Of course, this can
also be achieved with an indentation pattern.

\chapter{Numbers\label{NUMCHAP}}
\popasm\ accepts numbers in a variety of ways. A number is a sequence
of as many digits as you want\footnote{There is no practical limit to
number sizes in \popasm. If you want to write a 1,000,000-digits long
number, \popasm\ will accept it gladly (well, if you have patience to
type such a number...). Some other assemblers are limited to 32-bit
integer operations.} or separators, in any order, except that:

\begin{itemize}
\item{The first digit must be a 0--9 digit\footnote{Note that if the
   first digit is not decimal, \popasm\ (and other assemblers) will treat
   the string as a symbol, not a number. For instance, AH is a register,
   but 0AH is a number (10 in hex notation).}}
\item{All digits must be valid in the number base.} For example, ``3'' is
   not a valid digit in binary notation.
\end{itemize}

A separator is an underline (\_) character that may appear anywhere
among the number digits, except that it cannot be the first character.
They are ignored by \popasm, but help people read long numbers.

Examples:

\begin{itemize}
\item{12345678} is a valid number, because all of its digits are in the
   range 0--9
\item{12\_345\_678} is the same as above. Note how separators play the
   same role as a comma (that is, as a human being would write 12,345,678
   to make it more readable)
\item{12\_34\_5\_67\_\_\_8} also works, and is \emph{exactly} as the
   other two numbers above, despite looking weird and not being very useful.
\item{\_12\_345\_678} is not a number because the first character is not
   a digit.
\end{itemize}

Typical uses for separators include separating binary fields of bit
records. As an example, let's suppose that a simple communication
protocol uses a byte to encode information about a transmission as
follows:

\begin{table}[h]
\begin{center}
\begin{tabular}[h]{cl}
Bits & Description\\
\hline
6--7 & Package priority (0--3 range)\\
3--5 & Transmitter's ID (0--7 range)\\
0--2 & Receiver's ID (0--7 range)\\
\hline
\end{tabular}
\caption{Encoding of a simple transmission protocol header}
\end{center}
\end{table}
Thanks to the separators, the programmer can now write:

\begin{verbatim}
                         ; p  snd recv
                MOV     AL,11_000_010_B
\end{verbatim}

That is, a priority $p = 11 (binary) = 3$, a sender whose ID is $snd
= 000 (binary) = 0$ and a receiver whose ID is $recv = 010 (binary)
= 2$ are clearly encoded within AL. More complicated examples may
look messy without separators. The B suffix appended to that
number stands for ``binary'', as discussed in the next section.

\section{Using other bases}
As can be seen in the last example, \popasm\ allows one to specify a
number in many bases simply by appending a suffix. Bases and suffixes
supported by \popasm\ are:

\begin{table}[h]
\begin{center}
\begin{tabular}[h]{ccl}
Prefix & Base & Valid digits\\
\hline
b, B, y or Y & Binary (base 2)  & 0--1 \\
o, O, q or Q & Octal (base 8)  & 0--7 \\
d, D, t or T & Decimal (base 10)  & 0--9 \\
h or H & Hex (base 16)  & 0--9, A--F and a--f\protect{\footnotemark}\\
\hline
\end{tabular}
\caption{Prefix and bases}
\end{center}
\end{table}

\footnotetext{Note, however, that the first digit of an hex number
\emph{must} be in 0--9 range, as explained before.}

There are two other ways of declaring a number as hex:

\begin{itemize}
\item{Using the 0x or 0X prefix}, like in C language. If you use this
   prefix, the hex number that follows it does not need to begin with
   a decimal digit, that is, both 0xA and 0x0A are ok.
\item{Using the \$ prefix}, which may be dangerous. Mistaking -\$70 (-70h)
   with \$-70 (current offset minus 70) will surely lead to disaster.
\end{itemize}

It is important to note that if you do not explicitly specify a number
as binary, octal, decimal or hex, \popasm\ will use the current radix
value, which defaults to 10. That is, unless you specify otherwise, all
numbers will be read in decimal notation. See section \ref{RADIXCMD} if
you want to change the default radix.

\section{Real numbers}
Real numbers can be written in two forms:

\begin{itemize}
\item{The usual dot syntax}, like in 1.23, 3.1416, etc. Note that you can
use alternate bases to write real numbers as well. Eg.: 101.01B (5.25 in binary)
\item{Base and exponent}, as in 6.02e-23 (Avogadro\'{}s constant), 8.13e4 (8130),
etc.
\end{itemize}

Unlike some other assembers, \popasm\ can perform assembly-time operations
on both integer and real numbers. Real numbers have no storage limit either.
\popasm\ has also the advantage of neither rounding nor truncating real numbers.
They have their exact values stored as long as possible.

\chapter{Operators}
\popasm\ can perform several arithmetic operations on numbers and symbols. The
operators currently supported, in increasing order of precedence, are listed in
table \ref{OPTAB}.

\begin{table}[h]
\begin{center}
\begin{tabular}[h]{ccl}
\hline
Precedence & Operator & Description\\
\hline
                   & BYTE & 8-bit qualifier\\
                   & WORD & 16-bit qualifier\\
                   & DWORD & Double WORD --- 32-bit qualifier\\
                   & PWORD & triPle WORD --- 48-bit qualifier\\
                   & FWORD & Far WORD --- Same as PWORD\\
                   & QWORD & Quad WORD --- 64-bit qualifier\\
Lowest precedence  & OWORD & Oct WORD --- 128-bit qualifier\\
                   & TBYTE & Ten-BYTE --- 80-bit qualifier\\
                   & TWORD & Ten-WORD --- Same as TBYTE\protect{\footnotemark}\\
                   & SHORT & 8-bit relative displacement\\
                   & NEAR & 16- or 32-bit relative displacement\\
                   & FAR & 32- or 48-bit relative displacement\\
\hline
                   & OR $\vert$ & Inclusive or\\
                   & XOR \^{} & Exclusive or\\
\hline
                   & AND \& & Boolean AND\\
\hline
                   & + & Addition\\
                   & - & Subtraction\\
\hline
                   & * & Multiplication\\
                   & / & Division\\
                   & MOD & Remainer (modulus)\\
                   & SHL & Shift left\\
                   & SHR & Shift right\\
\hline
                   & NOT & One's complement\\
\hline
                   & : & Segment and offset composition\\
\hline
Highest precedence & . & Member selection in structs and unions\\
\hline
\end{tabular}
\caption{Operators and their precedence}
\label{OPTAB}
\end{center}
\end{table}

To be continued.

\footnotetext{Due to compatibility with \nasm.}

\chapter{Registers}
\popasm\ supports all registers present in each x86 CPU, except the ones
added in IA-64 architecture:

\begin{itemize}
\item{8-bit general-purpose registers:} AL, BL, CL, DL, AH, BH, CH and DH
\item{16-bit general-purpose registers:} AX, BX, CX, DX, SP, BP, SI and DI
\item{32-bit general-purpose registers:} EAX, EBX, ECX, EDX, ESP, EBP, ESI and EDI
\item{Segment registers:} CS, DS, ES, FS, GS and SS
\item{Control registers:} CR0 thru CR7
\item{Debug registers:} DR0 thru DR7
\item{Test registers:} TR0 thru TR7
\item{FPU registers:} ST(0) thru ST(7). ST0 thru ST7 are also accepted for
compatibility with \nasm. ST is an alias for ST(0).
\item{MMX registers:} MM0 thru MM7. MM is an alias for MM0.
\item{XMM registers:} XMM0 thru XMM7. XMM is an alias for XMM0.
\end{itemize}

Register names are case insensitive (i.e. AX, ax, aX and Ax are all the
same thing).

\chapter{Expressions\label{EXPR}}

An expression may be either a single term or a sequence of terms
connected by operators. A term, in turn, can be a number or a symbol.
This section explains how \popasm\ performs assembly-time arithmetics
and works with numbers.

\chapter{Memory references}
Memory references can be specified using square brackets [ ] around an expression.
For instance:

\begin{verbatim}
                MOV     AX,1234
\end{verbatim}

\noindent writes the value $1234$ in AX, but

\begin{verbatim}
                MOV     AX,[1234]
\end{verbatim}

\noindent writes the word pointed to by $1234$ in AX. In this case, DS is used as the
default segment register. If you wanted to use another segment register (say,
ES), you would write:

\begin{verbatim}
                MOV     AX,ES:[1234]
\end{verbatim}

\noindent or

\begin{verbatim}
                MOV     AX,[ES:1234]     ; same as above
\end{verbatim}

This latter form was added for compatibility with \nasm\ and \tasm\ ideal mode.
This document uses the former syntax throughoutly, though.

\section{Weak and strong memory references}
Programmers usually need to store data in variables. Those variables can then
be referenced by their names. For example, if VALUE is a word variable, its
contens can be copied to CX by the following command:

\begin{verbatim}
                MOV     CX,VALUE         ; CX = VALUE's contents
\end{verbatim}

Unfortunately, different assemblers may give different meanings to the line
of code above. \tasm\ behaves as described here, but \nasm\ would place the
variable's offset into CX. Because the name of a variable may be treated both
as its contents and its offset (depending on the assembler being used), such
memory references will be called \bf{weak memory references} in \popasm\ 
documentation.

\popasm\ behaves as described above (a variable's name means its contens, \bf{not}
its offset.) with respect to weak memory references. If you wish to refer to
a variable's offset, you should use the \emph{OFFSET} keyword. The example
above would be rewritten as:

\begin{verbatim}
                MOV     CX,OFFSET VALUE  ; CX = VALUE's offset
\end{verbatim}

On the other hand, a \bf{strong memory reference} is always enclosed within a
pair of matching square brackets. Example:

\begin{verbatim}
                MOV     CX,[VALUE]       ; CX = VALUE's contents
\end{verbatim}

The reader is strongly encouraged to avoid weak memory references, for its
meaning is assembler-dependant. Instead, whenever you refer to a variable's
offset, you should use the \emph{OFFSET} keyword, and whenever you want the
variable's contents, a strong memory reference is the best choice. Doing
that way, your code will rely neither on \popasm's default behavior, nor the
presence/absence of command-line options.

To be continued.

\chapter{\popasm\ commands}
\section{.RADIX\label{RADIXCMD}}
This command allows the developer to change the default radix of numbers
from the point the command is issued until the next .RADIX command or
end-of-file is found. \popasm\ sets the default radix to 10 at the start
of each pass.

\subsection{Syntax}

\begin{verbatim}
                .RADIX  expression
\end{verbatim}

\noindent where \emph{expression} is a constant integer numeric expression
that evaluates to 2, 8, 10 or 16. \emph{expression} cannot be affected by
.RADIX commands, that is, its default radix is always decimal regardless
of previous commands.

\subsection{Examples}

Our first example writes the same number in binary, using the ``B'' suffix
and the .RADIX command.

\begin{verbatim}
BINARY_NUMBER   DB      11000111B             ; 'B' means binary

                .RADIX  2
                ; Numbers default to be binary from now on

BINARY_NUMBER2  DB      11000111              ; same as above

                .RADIX  10
                ; Numbers default to be decimal again

DECIMAL_NUMBER  DD      11000111              ; decimal!
\end{verbatim}

It is still possible to use other bases when the default radix is not
decimal, but special care should be taken when numbers default to be hex.
In the next example, the developer changes the default radix to be 16
and attempts to define a variable whose value is 12 in decimal.

\begin{verbatim}
                .RADIX  16
                ; Numbers default to be hex from now on

AMBIGUOUS       DD      12D                   ; What does 'D' mean?
DECIMAL_NUMBER  DD      12T                   ; 'T' means decimal
HEX_NUMBER      DD      12                    ; 12h
\end{verbatim}

Note that 'D' is a valid digit in hex notation. This means that ``12D''
is interpreted as ``12Dh'', not 12 in decimal notation. In order to
avoid this ambiguity, \popasm\ also accepts ``T'' as a suffix for
decimal notation. It can be used no matter what the current radix is.
The same remarks hold for the binary suffixes ``B'' and ``Y'': ``11B''
is an hex number when the current radix is 16, but ``11Y'' is always
binary.

\begin{thebibliography}{12}
\bibitem{UFES} Federal University of Esp\'{\i}rito Santo -- ES -- Brazil.
http://www.ufes.br
\bibitem{PUC} Pontificial Catholic University of Rio de Janeiro -- RJ
-- Brazil. http://www.puc-rio.br
\bibitem{SF} SourceForge. http://sorceforge.net
\bibitem{ADV}Advogato. http://www.advogato.com
\bibitem{ADVPRSN}H\'elcio Mello's personal page at Advogato.
http://www.advogato.com/person/helcio
\bibitem{FRESH}FreshMeat. http://www.freahmeat.net
\bibitem{FREITAS}Professor PhD S\'ergio A. A. Freitas.
http://www.inf.ufes.br/\~{}sergio.
\bibitem{AUTOCONF}GNU Autoconf.
http://www.gnu.org/software/autoconf/autoconf.html
\bibitem{AUTOMAKE}GNU Automake.
http://www.gnu.org/software/automake/automake.html
\bibitem{MAKE}GNU Make. http://www.gnu.org/software/make/make.html
\end{thebibliography}

\end{document}
