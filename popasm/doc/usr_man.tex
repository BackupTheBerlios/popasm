\documentclass[a4paper,12pt]{book}
\title{PopAsm --- The Popular Assembler\\User's Manual}
\author{Helcio B. de Mello}
\date{\today}

\begin{document}
\maketitle

\chapter*{About this document}
This document is part of \emph{PopAsm, the Popular Assembler Project}. It has been totally written in \LaTeX\ by
\emph{PopAsm} author himself.

At the time of this writing, \emph{PopAsm} had two manuals:

\begin{itemize}
	\item{User's Manual} --- this document. It is aimed to those who wish to use \emph{PopAsm}, regardless the way
		it was developed and its internal implementation. This document also concerns about instalation procedures.
	\item{Programmer's Guide} --- documents \emph{PopAsm} source code in detail. It is recommended to those who
		want to read, understand, compile and/or modify \emph{PopAsm} sources.
\end{itemize}

Please check which manual is the one intended for your needs.

\chapter*{About the author}
H\'elcio Mello was born in Brazil, in 1979. He has a degree in Computer Engineering from Federal University of
Espirito Santo -- ES -- Brazil\cite{UFES}, and is curently taking Computer Science in Pontificial Catholic
University in Rio de Janeiro -- RJ -- Brazil\cite{PUC}. It is expected that he will be a Master of Science in 2004.

He joined \emph{Sourceforge}\cite{SF} in December 2001, and then registered \emph{PopAsm} as a \emph{SourceForge}
project. The project was finally open to the Free Software community. In July 2002, he joined \emph{Advogato}
\cite{ADV}\cite{ADVPRSN} and \emph{FreshMeat}\cite{FRESH}

\chapter*{Acknowledgements}
\emph{PopAsm} is the result of many months of work. Despite being the sole author of this project, I would like to
thank everybody that somehow contributed to this project. Among those are the hundreds of people who have already
visited \emph{PopAsm} home page, talked about it with friends, mailed me for comments, and so on.

I would also like to thank the ones who use \emph{PopAsm} and trust it. I know I can always count on you for
feedback about this project. \emph{Sourceforge} support was crucial, as it hosts \emph{PopAsm} CVS repository,
home page, and so on. Thank you very much.

Special thanks go to my first university, UFES\cite{UFES}, where I learned the bulk of my computering skills, and
to those classmates and teachers who believed \emph{PopAsm} project, such as professor and PhD S\'ergio Freitas%
\cite{FREITAS}, who decided to adopt \emph{PopAsm} as the assembler for his graduate classes about assembly
language.

Last, but not least, I would like to thank my family, friends and God for what I am now and what I have done so far.

\chapter{Introduction}
\emph{PopAsm} stands for ``Popular Assembler''. It is an assembler, that is, and assembly language compiler\footnote{Technically
speaking, assemblers are not compilers. They translate lines of code into machine language in a one-to-one basis. Compilers
translate each line of code into several machine instructions. However, from this point on, this document will use both
terms (compiler and assembler) indistinctly.}. Its objective is to convert human readable code to machine instructions.
These instructions will then be either executed as binary programs or linked with other modules (possibly written in
other languages) to yield a computer program.

Many assemblers already exist. Some are free and open source, others are not. Some will offer you features that others
will not. Some will be suited for your needs, but others will not. \emph{PopAsm} was designed to gather in a single
assembler the best features of the existing assemblers, yet adding its own improvements and remaining compatible with
existing code as well. As a result, most of your legacy code can be compiled under \emph{PopAsm} without any modifications
at all.

Besides the benefits discussed so far, \emph{PopAsm} is a free open source project written in ANSI C++, which means
that anyone can read its source code, modify, and compile it anywhere an ANSI C++ compiler is avaiable. Its peculiar
features make it suited for nearly any assembly programming project:

\begin{itemize}
	\item{Huge numbers internal representation} allows assembly-time expression evaluation in both integer and floating
		point format without any practical limit;
   \item{Smart default options} make your code cleaner, without the redundancy demanded by some other assemblers;
	\item{Top flexibility} gives the developers the choice to use the infamous ``red tape'' present in some assemblers
		or just write the good old raw assembly code;
	\item{And more...}
\end{itemize}

Due to the reasons discussed above, this assembler was named the ``Popular Assembler''.

\chapter{Compiling and Installing PopAsm}
This chapter explains how to compile and install \emph{PopAsm} under linux or other UNIX-like
platforms. If you have already done that you may safely skip to the next chapter.

\section{Compiling PopAsm sources\label{COMPILING}}
In order to compile \emph{PopAsm} sources, you will need an ANSI C++ compiler and its standard
libraries\footnote{For the curious, I use egcs 0.0.0}. You will also need \emph{PopAsm} source
code, which you should already have. If not, please go to \emph{http://popasm.sourceforge.net/}
and download it. You will get an archive containing the source code to be built. Unpack it
anywhere you like, using the appropriate software (e.g. if you downloaded a .tar.gz file, you
should run \emph{tar xvzf popasm-x.y.z.tar.gz}, where $x$, $y$ and $z$ are the version numbers
of the package you got). At this point, a directory containing the source code will be created.

Now, you should enter the directory containing the sources. e.g. \emph{/tmp/popasm-0.0.1}. Please
do not mistake it for something like \emph{/tmp/popasm-0.0.1/src}. You should be in \emph{PopAsm}
source code root directory, as in the example above.

\emph{PopAsm} relies on \emph{Autoconf}\cite{AUTOCONF} and \emph{Automake}\cite{AUTOMAKE} to probe
your system for the necessary resources and generate the resulting \emph{Makefile}. This file will
tell the \emph{Make}\cite{MAKE} utility how to compile \emph{PopAsm} sources. First, type

\begin{verbatim}
./configure
\end{verbatim}

to perform the necessary checks. If everything is ok, you should have a \emph{Makefile}. The next
step is to compile the sources. Just type

\begin{verbatim}
make
\end{verbatim}

and the \emph{Make} utility will do the rest, but might take a few minutes. There should be no
warnings and no errors. If you got any, please check whether your compiler is ANSI compliant. If
it is, please let me know (contacting information can be found in this document).

\section{Installing PopAsm}

There is basically two ways of installing \emph{PopAsm}: either from the source code or from a
RPM file.

\subsection{Installing from Source Code}
You should compile \emph{PopAsm} as in section \ref{COMPILING}. Remain in the directory you built
\emph{PopAsm} and issue the command (you will need to be \emph{root} to do that):

\begin{verbatim}
make install
\end{verbatim}

This will install the binary file, documentation, etc.

\subsection{Installing from RPM files}
The easiest way to install \emph{PopAsm} is to use RPM files. Get one from
\emph{http://popasm.sourceforge.net/} if you do not have done so yet. Login as \emph{root} and
type:

\begin{verbatim}
rpm -ivh popasm-x.y.z.rpm
\end{verbatim}

replacing $x$, $y$ and $z$ for the appropriate values. That's it, you're done.

\begin{thebibliography}{9}
\bibitem{UFES} Federal University of Esp\'irito Santo -- ES -- Brazil. http://www.ufes.br
\bibitem{PUC} Pontificial Catholic University of Rio de Janeiro -- RJ -- Brazil. http://www.puc-rio.br
\bibitem{SF} Pontificial Catholic University of Rio de Janeiro -- RJ -- Brazil. http://www.puc-rio.br
\bibitem{ADV}Advogato. http://www.advogato.com
\bibitem{ADVPRSN}H\'elcio Mello's personal page at Advogato. http://www.advogato.com/person/helcio
\bibitem{FRESH}FreshMeat. http://www.freahmeat.net
\bibitem{FREITAS}Professor PhD S\'ergio A. A. Freitas. http://www.inf.ufes.br/~sergio.
\bibitem{AUTOCONF}GNU Autoconf. http://www.gnu.org/software/autoconf/autoconf.html
\bibitem{AUTOMAKE}GNU Automake. http://www.gnu.org/software/automake/automake.html
\bibitem{MAKE}GNU Make. http://www.gnu.org/software/make/make.html
\end{thebibliography}

\end{document}
